\documentclass[aspectratio=169]{beamer}

\input{FMCS-preamble}

\definecolor{darkMagenta}{HTML}{ff9900}
\definecolor{violet}{HTML}{ff9900}
\definecolor{teal}{HTML}{1260cc}

\usepackage{tabularx}
\usepackage{xcolor,colortbl}
\usepackage{ cmll }
\usepackage{panqueques}

\makeatletter
\def\blfootnote{\gdef\@thefnmark{}\@footnotetext}
\makeatother

\newcommand{\purple}[1]{\textcolor{purple}{#1}} 
\newcommand{\tcolor}[1]{\textcolor{magenta}{#1}}

\title{What kind of linearly distributive categories \\ do polynomial functors form?}
\author{Priyaa Varshinee Srinivasan}
\date{\today}

\begin{document} 

\begin{frame}[noframenumbering,plain]
    \begin{tikzpicture}[remember picture, overlay]
        \fill[black] (current page.south west) rectangle ([xshift=2cm]current page.north west);
    \end{tikzpicture}
    \begingroup  
        %\flushright
        \centering
        {\fontfamily{qag}\selectfont\large\bfseries\color{black}\MyTitle} 
        \par \vspace{1em} \MyAuthor  
        \par \par \vspace{1em} \includegraphics[scale=0.5]{pics/Topos.png}
        \par \vspace{-1em} \small{Topos Institute, Berkeley}
        \par \vspace{2em} \small{FMCS 2024, Kananaskis}
        \par \vspace{-0.5em} 
        \small{\MyDate}\par
    \endgroup
\end{frame}

\begin{frame}[noframenumbering, plain]

    \begin{tikzpicture}[remember picture, overlay]
        \fill[black] (current page.south west) rectangle ([xshift=2cm]current page.north west);
    \end{tikzpicture}
    \begingroup  

{\centering
This is joint work with \textcolor{magenta}{David Spivak}. 

\[ \includegraphics[scale=0.35]{pics/hqdefault.jpg} \]

Our paper is available in \textcolor{magenta}{arXiv: 2407.01849} (July 2024) 

}

\endgroup

\end{frame}

\begin{frame}[noframenumbering,plain]

\begin{tikzpicture}[remember picture, overlay]
        \fill[black] (current page.south west) rectangle ([xshift=2cm]current page.north west);
    \end{tikzpicture}
    \begingroup  
        \flushleft
        {\fontfamily{qag}\selectfont\hspace{2 cm}\Large\bfseries\color{black}{Linearly distributive categories}}\vspace{1em}\\
        \hspace{2 cm} Robin Cockett, and Robert Seely. {\em Weakly distributive categories} (1997) \\
    \endgroup
\end{frame}

\iffalse

\begin{frame}{Linear logic}

In 1987, Girard introduced linear logic as a logic for resources manipulation.

\vspace{0.5em}

Classical logic treats statements as truth values; linear logic treats statements as resources which cannot be duplicated or destroyed.

\vspace{-1em}

\begin{align*}
    \textcolor{magenta}{p} &:  \text{to spend a dollar} \\
    \textcolor{magenta}{q} &: \text{to buy an apple} 
\end{align*}

``$p \Rightarrow q$" has the meaning that if a dollar is spent then an apple 
can be bought. 

\vspace{0.5em}

A person can either have \textcolor{magenta}{a dollar} or \textcolor{magenta}{an apple} at a given time but not both. 

\vspace{0.5em}
    
    The word ``linear" refers to this resource sensitivity of the 
    logic.
\end{frame}

\begin{frame}{Categorical semantics of multiplicative linear logic}

{\small
      \begin{table}[h]
        \centering
    	\begin{tabular}{ | l | l | }
        \hline
    	 {\bf Linear logic fragment} & {\bf Connectives}  \\  
       \hline 
       \hline
      Negation & $A^\perp$ \\  
      \hline
    	\cellcolor{red!20}Multiplicative  & \cellcolor{red!20}  $(\ox, 1)$ and $(\parr, \bot)$ \\ 
      \hline
    	Additive & $(\with, \top)$ and $(\oplus, 0)$ \\ 
      \hline
    	Exponentials & $!$ and $?$ \\ 
      \hline
    	\end{tabular}
    \end{table}


}

{\bf Negation}: 

\textcolor{magenta}{p}: to spend a dollar \qquad \qquad 
\textcolor{magenta}{$p^\perp$}: to receive a dollar 

    \vspace{0.25em}

  {\bf Multiplicative fragment}: 

    $(\ox, 1)$: an apple $\ox$ an orange ($p \Rightarrow q \ox r$ means access to resources at the same time; conjunction from classical logic)

    $(\parr, \bot)$ = De Morgan's Rule: $(A^\perp \ox B^\perp)^\perp$ (means don't have access to either resource, so someone else owns it; disjunction from classical logic)

\end{frame}


\begin{frame}{Categorical semantics of multiplicative linear logic}

{\small
      \begin{table}[h]
        \centering
    	\begin{tabular}{ | l | l | }
        \hline
    	 {\bf Linear logic fragment} & {\bf Connectives}  \\  
       \hline 
       \hline
      Negation & $A^\perp$ \\  
      \hline
    	\cellcolor{red!20}Multiplicative  & \cellcolor{red!20}  $(\ox, 1)$ and $(\parr, \bot)$ \\ 
      \hline
    	Additive & $(\with, \top)$ and $(\oplus, 0)$ \\ 
      \hline
    	Exponentials & $!$ and $?$ \\ 
      \hline
    	\end{tabular}
    \end{table}


    \begin{table}[h]
    \centering
	\begin{tabular}{ | l | l | }
    \hline
	 \cellcolor{violet!20}{\bf Linear logic fragment} & \cellcolor{violet!20}{\bf Categorical proof theory}  \\  
   \hline 
   \hline
     \cellcolor{red!20}{MLL} & \cellcolor{red!20}{Linearly distributive categories\footnote{Cockett and Seely (1997) ``Weakly Distributive Categories"} }  \\
   	 \cellcolor{red!20}{ $(\ox, 1)$ and $(\parr, \bot)$ }& \cellcolor{red!20}{$(\X, \ox, \top, \oa, \bot)$}\\
  \hline
	 MLL with negation & $*$-autonomous categories\footnote{Barr (1991) ``$*$-autonomous categories and linear logic"} \\ 
  \hline
    \cellcolor{red!20} Compact MLL &  \cellcolor{red!20} Monoidal categories  \\ 
	 \cellcolor{red!20}($\otimes = \parr$, $1 = \bot$) &  \cellcolor{red!20} $(\X, \ox, I)$  \\
  \hline
	Compact closed categories\footnote{Kelly and Laplaza (1980) ``Coherences for compact closed categories"} & Compact MLL with negation \\ 
  \hline
	\end{tabular}
    \label{Table: MLL} 
\end{table}
}

\end{frame}

\fi

\begin{frame}{Linearly distributive categories}

 {\bf Linearly distributive categories (LDCs)\footnote{Cockett and Seely (1997) ``Weakly distributive categories"}:}
    \[
    (\X, \ox, \top, a_\ox, u_\ox^L, u_\ox^R) ~~~~~~~~~ (\X, \oa, \bot, a_\oa, u_\oa^L, u_\oa^R)
    \]  linked by {linear distributors}: 
    \[\partial^L: A \ox (B \oa C) \rightarrow  (A \ox B) \oa C \]
    \[  \textcolor{gray} {A \times (B + C) \simeq (A \times B) + (A \times C)} \]
    \[ \partial^R: (B \oa C) \ox A \rightarrow  B \oa (C \ox A) \]

    \textbf{Intuition:} At a restaurant the waiter, A, can choose to address either person at the table, B or C. Once assigned to B, A cannot choose C.

    \textit{The distributor is not an equality or isomorphism in general!}
        
    {\bf Monoidal categories:} LDCs in which $\ox = \oa; \top = \bot$ 

    \vspace{1em}
\end{frame}

\begin{frame}{Symmetry and $\ox$-symmetry}

A \tcolor{symmetric LDC} is an LDC in which $\ox$ and the $\oa$ products are symmetric and the diagram commutes.
\[ \begin{tikzcd}[ampersand replacement=\&]
(A \oa B) \ox C \ar[d, "\tcolor{\partial^R}"'] \ar[r, "\sigma_\ox" ]
	\& C \ox (A \oa B) \ar[r, "\sigma_\oa"]  
	\& C \ox (B \oa A) \ar[d, "\tcolor{\partial^L}"] \\ 
A \oa (B \ox C) 
	\& A \oa (C \ox B)  \ar[l, "\sigma_\oa"]
	\& A \oa (B \ox C) \ar[l, "\sigma_\ox"] 
\end{tikzcd}\]

\vspace{1em}

A \tcolor{$\ox$-symmetric LDC} is an LDC in which only the $\ox$ product is symmetric.

\end{frame}

\begin{frame}{Spectrum of LDCs}

        \vspace{0.5em}
    
  \[  \begin{tikzpicture}[scale=2]
        \begin{pgfonlayer}{nodelayer}
            \node [style=circle, scale=2, color=magenta, fill=magenta] (0) at (-5.75, 2.75) {};
            \node [style=circle, scale=2, color=black!80, fill=black!80] (1) at (-3.5, 2.75) {};
            \node [style=circle, scale=2, color=black!60, fill=black!60] (2) at (-1, 2.75) {};
            \node [none, color=white] (99) at (-3.5, 2.75) {?};
            \node [style=circle, scale=2, color=black!40, fill=black!40] (3) at (1.75, 2.75) {};
            \node[style=none, color=white] (99) at (1.75, 2.75) {?};
            \node[style=none, color=white] (99) at (-1, 2.75) {?};
            \node [style=none] (4) at (-7.75, 2.75) {};
            \node [style=circle, scale=2, color=magenta!20, fill=magenta!20] (5) at (4, 2.75) {};
            \node [style=none] (6) at (6, 2.75) {};
            \node [style=none] (7) at (-5.75, 2) {{\large \bf LDC}};
            \node [style=none] (8) at (-3.5, 4.35) {};
            \node [style=none] (9) at (-3.5, 3.85) {};        
            \node [style=none] (10) at (-1, 2) {};
            \node [style=none] (11) at (1.75, 4.1) {};
            \node [style=none] (12) at (1.75, 3.5) {};
            \node [style=none] (13) at (4, 2) {};
            \node [style=none] (14) at (-1, 1.4) {};
            \node [style=none] (15) at (4, 1.5) {};
            \node [style=none] (16) at (-5.75, 1.5) {$(\X, \ox, \top, \oa, \bot)$};
            \node [style=none] (17) at (-3.5, 3.35) { };
            \node [style=none] (13) at (4, 2) {{\bf \large Monoidal} category};
            \node [style=none] (15) at (4, 1.5) {$(\X, \ox, I)$};
        \end{pgfonlayer}
        \begin{pgfonlayer}{edgelayer}
            \draw [dotted] (4.center) to (0);
            \draw (0) to (1);
            \draw (1) to (2);
            \draw (2) to (3);
            \draw (3) to (5);
            \draw [dotted] (5) to (6.center);
        \end{pgfonlayer}
    \end{tikzpicture}\]
    
\end{frame}

\begin{frame}{Mix LDCs}

{\bf Mix category} \footnote{Richard Blute, Robin Cockett, and Robert Seely (2000). "Feedback for linearly distributive categories: traces and fixpoints."}: LDC with $m: \bot \to \top$ called the {\bf mix map} with

\begin{center}
$\indep_{A,B}: A \ox B \to A \oa B :=$
\begin{tikzpicture}
	\begin{pgfonlayer}{nodelayer}
		\node [style=ox] (0) at (0, 0.2500001) {};
		\node [style=circ] (1) at (0.5000001, -0.2500001) {};
		\node [style=circ] (2) at (0, -1) {$\bot$};
		\node [style=map] (3) at (0, -1.75) {m};
		\node [style=circ] (4) at (0, -2.5) {$\top$};
		\node [style=circ] (5) at (-0.5000001, -3.25) {};
		\node [style=oa] (6) at (0, -3.75) {};
		\node [style=nothing] (7) at (0, 0.7499999) {};
		\node [style=nothing] (8) at (0, -4.25) {};
	\end{pgfonlayer}
	\begin{pgfonlayer}{edgelayer}
		\draw (7) to (0);
		\draw (0) to (1);
		\draw [in=45, out=-60, looseness=1.00] (1) to (6);
		\draw [in=120, out=-135, looseness=1.00] (0) to (5);
		\draw (5) to (6);
		\draw (6) to (8);
		\draw [densely dotted, in=-90, out=45, looseness=1.00] (5) to (4);
		\draw (4) to (3);
		\draw (3) to (2);
		\draw [densely dotted, in=-135, out=90, looseness=1.00] (2) to (1);
	\end{pgfonlayer}
\end{tikzpicture}
=
\begin{tikzpicture}
	\begin{pgfonlayer}{nodelayer}
		\node [style=circ] (0) at (-0.5000001, -0.2500001) {};
		\node [style=circ] (1) at (0, -1) {$\bot$};
		\node [style=map] (2) at (0, -1.75) {m};
		\node [style=circ] (3) at (0, -2.5) {$\top$};
		\node [style=circ] (4) at (0.5000001, -3.25) {};
		\node [style=nothing] (5) at (0, 0.7499999) {};
		\node [style=nothing] (6) at (0, -4.25) {};
		\node [style=oa] (7) at (0, -3.75) {};
		\node [style=ox] (8) at (0, 0.2500001) {};
	\end{pgfonlayer}
	\begin{pgfonlayer}{edgelayer}
		\draw [densely dotted, in=-90, out=150, looseness=1.00] (4) to (3);
		\draw (3) to (2);
		\draw (2) to (1);
		\draw [densely dotted, in=-45, out=90, looseness=1.00] (1) to (0);
		\draw (8) to (5);
		\draw (8) to (0);
		\draw [in=135, out=-120, looseness=1.00] (0) to (7);
		\draw (7) to (6);
		\draw (7) to (4);
		\draw [in=-45, out=60, looseness=1.00] (4) to (8);
	\end{pgfonlayer}
\end{tikzpicture} 
\[ (1 \oa (u_\oa^L)^{-1}) (1 \ox (\m \oa 1)) \delta^L(u_\ox^R \oa 1)\]
\end{center}
The \tcolor{indep}\footnote{Also referred to as the mixor} must be natural in $A$ and $B$.

\vspace{0.5em}

\end{frame}

\begin{frame}[noframenumbering]{Spectrum of LDCs}

\[
    \begin{tikzpicture}[scale=2]
        \begin{pgfonlayer}{nodelayer}
            \node [style=circle, scale=2, color=magenta, fill=magenta] (0) at (-5.75, 2.75) {};
            \node [style=circle, scale=2, color=magenta!80, fill=magenta!80] (1) at (-3.5, 2.75) {};
            \node [style=circle, scale=2, color=black!60, fill=black!60] (2) at (-1, 2.75) {};
            \node [style=circle, scale=2, color=black!40, fill=black!40] (3) at (1.75, 2.75) {};
            \node [style=none] (4) at (-7.75, 2.75) {};
            \node [style=circle, scale=2, color=magenta!20, fill=magenta!20] (5) at (4, 2.75) {};
            \node [style=none] (6) at (6, 2.75) {};
            \node [style=none] (7) at (-5.75, 2) {{\large \bf LDC}};
            \node [style=none] (8) at (-3.5, 4.35) {{\bf \large Mix} category};
            \node [style=none] (9) at (-3.5, 3.85) {$\m: \bot \to \top$};        
            \node [style=none] (10) at (-1, 2) {};
            \node [style=none] (11) at (1.75, 4.1) {};
            \node [style=none] (12) at (1.75, 3.5) {};
            \node [style=none] (13) at (4, 2) {};
            \node [style=none] (14) at (-1, 1.4) {};
            \node [style=none] (15) at (4, 1.5) {};
            \node [style=none] (16) at (-5.75, 1.5) {$(\X, \ox, \top, \oa, \bot)$};
            \node [style=none] (17) at (-3.5, 3.35) {$\mx: A \ox B \to A \oa B$};
            \node [style=none] (13) at (4, 2) {{\bf \large Monoidal} category};
            \node [style=none] (15) at (4, 1.5) {$(\X, \ox, I)$};
            \node[style=none, color=white] (99) at (1.75, 2.75) {?};
            \node[style=none, color=white] (99) at (-1, 2.75) {?};
        \end{pgfonlayer}
        \begin{pgfonlayer}{edgelayer}
            \draw [dotted] (4.center) to (0);
            \draw (0) to (1);
            \draw (1) to (2);
            \draw (2) to (3);
            \draw (3) to (5);
            \draw [dotted] (5) to (6.center);
        \end{pgfonlayer}
    \end{tikzpicture}
\]
    
\end{frame}

\begin{frame}{Isomix LDCs }

\vspace{0.5em}

    It is an {\bf isomix} category if $m$ is an isomorphism.  

\vspace{1em}

    $m$ being an isomorphism does not make the indep an isomorphism.

\vspace{1em}

    \[ \begin{tikzpicture}[scale=2]
        \begin{pgfonlayer}{nodelayer}
            \node [style=circle, scale=2, color=magenta, fill=magenta] (0) at (-5.75, 2.75) {};
            \node [style=circle, scale=2, color=magenta!80, fill=magenta!80] (1) at (-3.5, 2.75) {};
            \node [style=circle, scale=2, color=magenta!60, fill=magenta!60] (2) at (-1, 2.75) {};
            \node [style=circle, scale=2, color=black!40, fill=black!40] (3) at (1.75, 2.75) {};
            \node [style=none] (4) at (-7.75, 2.75) {};
            \node [style=circle, scale=2, color=magenta!20, fill=magenta!20] (5) at (4, 2.75) {};
            \node [style=none] (6) at (6, 2.75) {};
            \node [style=none] (7) at (-5.75, 2) {{\large \bf LDC}};
            \node [style=none] (8) at (-3.5, 4.35) {{\bf \large Mix} category};
            \node [style=none] (9) at (-3.5, 3.85) {$\m: \bot \to \top$};
            \node [style=none] (10) at (-1, 2) {{\bf \large Isomix} category};
            \node [style=none] (13) at (4, 2) {{\bf \large Monoidal} category};
            \node [style=none] (14) at (-1, 1.4) {$\bot \to^{\m}_{\simeq} \top$};
            %\node [style=none] (15) at (4, 1.5) {$\m = 1$, $\mx=1$};
            \node [style=none] (16) at (-5.75, 1.5) {$(\X, \ox, \top, \oa, \bot)$};
            \node [style=none] (17) at (-3.5, 3.35) {$\mx: A \ox B \to A \oa B$};
            \node [style=none] (15) at (4, 1.5) {$(\X, \ox, I)$};
            \node[style=none, color=white] (99) at (1.75, 2.75) {?};
        \end{pgfonlayer}
        \begin{pgfonlayer}{edgelayer}
            \draw [dotted] (4.center) to (0);
            \draw (0) to (1);
            \draw (1) to (2);
            \draw (2) to (3);
            \draw (3) to (5);
            \draw [dotted] (5) to (6.center);
        \end{pgfonlayer}
    \end{tikzpicture} \]
\end{frame} 


\begin{frame}{Compact LDCs}

A {\bf compact LDC} is an LDC in which every indep map is an isomorphism. 
\[ \indep_{A,B}: A \ox B \to^{\simeq}  A \oa B \]

\vspace{1.5em}
 
Compact LDCs $(\X, \ox, \top, \oa, \bot)$ are linearly equivalent to the underlying 
monoidal categories $(\X, \ox, \top)$ and $(\X, \oa, \bot)$.    
\end{frame}

 \begin{frame}{Spectrum of LDCs}

    \[    \begin{tikzpicture}[scale=2]
        \begin{pgfonlayer}{nodelayer}
            \node [style=circle, scale=2, color=magenta, fill=magenta] (0) at (-5.75, 2.75) {};
            \node [style=circle, scale=2, color=magenta!80, fill=magenta!80] (1) at (-3.5, 2.75) {};
            \node [style=circle, scale=2, color=magenta!60, fill=magenta!60] (2) at (-1, 2.75) {};
            \node [style=circle, scale=2, color=magenta!40, fill=magenta!40] (3) at (1.75, 2.75) {};
            \node [style=none] (4) at (-7.75, 2.75) {};
            \node [style=circle, scale=2, color=magenta!20, fill=magenta!20] (5) at (4, 2.75) {};
            \node [style=none] (6) at (6, 2.75) {};
            \node [style=none] (7) at (-5.75, 2) {{\large \bf LDC}};
            \node [style=none] (8) at (-3.5, 4.35) {{\bf \large Mix} category};
            \node [style=none] (9) at (-3.5, 3.85) {$\m: \bot \to \top$};
            \node [style=none] (10) at (-1, 2) {{\bf \large Isomix} category};
            \node [style=none] (11) at (1.75, 4.1) {{\bf \large Compact} LDC};
            \node [style=none] (12) at (1.75, 3.5) {$A \ox B \to^{\mx}_{\simeq} A \oa B$};
            \node [style=none] (13) at (4, 2) {{\bf \large Monoidal} category};
            \node [style=none] (14) at (-1, 1.4) {$\bot \to^{\m}_{\simeq} \top$};
            \node [style=none] (15) at (4, 1.5) {$\m = 1$, $\mx=1$};
            \node [style=none] (16) at (-5.75, 1.5) {$(\X, \ox, \top, \oa, \bot)$};
            \node [style=none] (17) at (-3.5, 3.35) {$\mx: A \ox B \to A \oa B$};
        \end{pgfonlayer}
        \begin{pgfonlayer}{edgelayer}
            \draw [dotted] (4.center) to (0);
            \draw (0) to (1);
            \draw (1) to (2);
            \draw (2) to (3);
            \draw (3) to (5);
            \draw [dotted] (5) to (6.center);
        \end{pgfonlayer}
    \end{tikzpicture} \]
    
\end{frame}

\begin{frame}{Typical examples of LDCs}

\vspace{2em}

Every monoidal category is an LDC.

\vspace{0.75em}

A bounded distributive lattice regarded as a category is an LDC.

\vspace{0.75em}

All $*$-autonomous categories are LDCs: \\
Ehrhard’s finiteness spaces, Girard's coherence spaces, Chu spaces.

\vspace{1.5em}

\begin{center} 

\textcolor{purple}{\Large Can we have a simple yet a structurally rich example of LDCs?}

\vspace{0.5em}

{\Large Yes!!}

\end{center}

\end{frame}


\begin{frame}{This talk is about ... }

\vspace{1em}

\textcolor{purple}{Poly} the category of polynomial functors and transformations as an example of isomix LDCs.

\vspace{1em}

We will find non-trivial examples of various {structures of LDCs} in Poly.

\vspace{1em}

We will also {discover new properties} of LDCs by studying Poly.

\vspace{2em}

\begin{center} \textcolor{magenta}{Poly seems to be a golden goose for LDCs!!!} \end{center}
   
\end{frame}

\begin{frame}{Acknowledgement}
    \centering
    
    Thanks to Harrison Grodin and Reed Mullanix for their conjecture, that normal duoidal categories are linear distributive categories, which spawned this work.
    
\end{frame}

\begin{frame}{ Recap: Polynomial functors and natural transformations}

\vspace{1em}

A {\bf polynomial functor} is a functor that is isomorphic to a coproduct of representables.

 \[ p \cong \sum_{P : p(1)} y^{p[P]} : {\sf Set \to Set} \] 

\vspace{0.5em}

A schematic of a polynomial functor map, $\varphi: (y^3 + y^2) \to (y + y^2)$.
\[ \includegraphics[scale=0.09]{pics/corolla-map.jpeg} \] 
\end{frame}

\begin{frame}{ Recap: Tensor product in Poly}

The tensor product $\ox$ is given by the Day convolution of Cartesian product $\times$ in $\Set$.  

\[ p \ox q = \sum_{P: p(1)} y^{p[P]} \ox \sum_{Q: q(1)} y^{q[Q]} := \sum_{\textcolor{purple}{(P,Q): p(1) \times q(1)}} y^{\textcolor{purple}{p[P] \times q[Q]}} \] 

Tensor product is symmetric.

\end{frame}


\begin{frame}{Recap: Tri product in Poly}

The substitution product $\tri$ is given by functor composition. 

\[ p \tri q = \sum_{P: p(1)} y^{p[P]} \tri \sum_{Q: q(1)} y^{q[Q]} :=  
 \sum_{P: p(1)} \Big(\sum_{Q: q(1)} y^{q[Q]} \Big)^{p[P]} 
 = \sum_{P:p(1)} \sum_{\color{purple}{f:p[P] \to q(1)}} \prod_{d:p[P]} \prod_{\color{purple}{e:q[f(d)]}} y \]
 
Read $p \tri q$ as \textcolor{purple}{q then p}.

\vspace{1em}

The tri product $\tri$ is non-symmetric.
\[ \includegraphics[scale=0.08]{pics/tri.jpeg} \]
\end{frame}

\begin{frame}[noframenumbering,plain]

\begin{tikzpicture}[remember picture, overlay]
        \fill[black] (current page.south west) rectangle ([xshift=4cm]current page.north west);
    \end{tikzpicture}
    \begingroup  
        \flushleft
        {\fontfamily{qag}\selectfont\hspace{2 cm}\Large\bfseries\color{black}{Part I: The category Poly is a $\ox$-symmetric isomix LDC}} \vspace{1em}
         
    \endgroup
\end{frame}

\begin{frame}{Duoidal categories}

   A \tcolor{duoidal category}\footnote{Marcelo Aguiar and Swapneel Arvind Mahajan. Monoidal functors, species and Hopf algebras (2010)} is a category $\X$ with two monoidal structures $(\X, \ox, \top)$ and $(\X, \oa, \bot)$ along with a natural transformation:
    \[ {\duo: ~(A \oa B) \ox (C \oa D) \to (A \ox C) \oa (B \ox D)} \]
    called the \tcolor{interchange law}, and morphisms:
    \[ e_\top: ~\top \to \top \oa \top \quad \quad  e_\bot: ~\bot \otimes \bot \to \bot \]    
    such that the functors $\oa$ and $\bot$ are $\ox$-lax monoidal, and the assosciativity and unitor natural isomorphisms of $(\oa, \bot)$ are $\ox$-monoidal natural transformations.
\vspace{1em}

\end{frame}

\begin{frame}{Normal duoidal category}
 
 In a duoidal category, we have a map $\tcolor{k: \top \to \bot}$

\[ \top \to^{\cong}  \top \ox \top \to^{\cong}  (\top \oa \bot) \ox (\bot \oa \top) \to^{\tcolor{\duo}} (\top \ox \bot) \oa (\top \ox \bot) \to^{\cong} \bot \oa \bot \to^{\cong} \bot \] 

\vspace{0.5 em}

A duoidal category is \tcolor{normal} if the above composite is an isomorphism.

\end{frame}

\begin{frame}{Bilax duoidal functor}

\tcolor{Bilax duoidal functor} $F: (\X, \ox, \top, \tri, \bot) \to (\Y, \ox, \top, \tri, \bot)$ consists of a functor $F: \X \to \Y$ that is 

$\ox$-lax monoidal and  $\tri$-colax monoidal satisfying,
\[ \xymatrix{
        F((a \tri b) \otimes (c \tri d)) \ar[d]_{F(\duo)} &F(a \tri b) \otimes F(c \tri d) \ar[r]^{\hspace{-20pt}n_\tri \otimes n_\tri} \ar[l]_{m_\otimes} & (F(a) \tri F(b)) \otimes (F(c) \tri F(d)) \ar[d]^{\duo} \\ 
        F((a \otimes c) \tri (b \otimes d)) \ar[r]_{n_\tri} &F(a \otimes c) \tri F(b \otimes d) & (F(a) \otimes F(c)) \tri (F(b) \otimes F(d)) \ar[l]^{\hspace{-20pt}m_\otimes \tri m_\otimes} 
    }
 \]
 and 3 more coherences (two for unit laxors the duoidal map and one for the unit laxors with the $k$ map.

\end{frame}

\begin{frame}{Normal duoidal is also isomix}

{\bf normalDuo} category of normal duoidal categories and bilax duoidal functors

\vspace{0.5em}

{\bf Isomix} category of isomix LDCs and isomix functors

\vspace{1em}

\textcolor{teal}{Theorem:} There is a faithful functor from \tcolor{normalDuo} to \tcolor{Isomix}.

\textcolor{teal}{Proof sketch:} Define the left distributor as follows:
\[ 
\begin{tikzcd}[ampersand replacement=\&]
A \ox (B \oa C) \ar[r, "\color{magenta}\cong"] \& (A \oa I) \ox (B \oa C) \ar[r, "\color{magenta}\duo"] \& (A \ox B) \oa (I \ox C) \ar[r, "\color{magenta}\cong"] \& (A \ox B) \oa C 
\end{tikzcd}
\]

\vspace{0.75em}

 \textcolor{teal}{Lemma}: The category \tcolor{$($Poly$, \otimes, \tri,  y )$ is normal duoidal} hence an isomix LDC. Additionally, Poly is $\ox$-symmetric. 
     
\vspace{0.5em}

\color{violet}{\bf Attention:} \color{black} Since in this tutorial we will mostly be in a $\ox$-symmetric setting, we will \\ \tcolor{use $\tri$ instead of $\oa$}.    
    
\end{frame}


\begin{frame}[noframenumbering,plain]
\begin{tikzpicture}[remember picture, overlay]
        \fill[black] (current page.south west) rectangle ([xshift=4cm]current page.north west);
    \end{tikzpicture}
    \begingroup  
        \flushleft
        {\fontfamily{qag}\selectfont\hspace{2 cm}\Large\bfseries\color{black}{Part II: Meet the linear duals and biclosed LDCs}}\vspace{1em}\\
    \endgroup
\end{frame}

\begin{frame}{What is a linear dual?}
    
    In an LDC, an object {\bf $B$ is left dual to $A$} if there exist\footnote{Robin Cockett, Jurgen Koslowski and Robert Seely. Introduction to Linear bicategories (1999)}:
    \[ \eta: \top \to A \tri B ~~~~~~~~ \epsilon: B \ox A \to \bot \]
    such that: 
     \[	\begin{tikzpicture}[scale=1.2]
    			\begin{pgfonlayer}{nodelayer}
    				\node [style=none] (6) at (1, 0) {};
    				\node [style=none] (7) at (1, 1) {};
    				\node [style=none] (8) at (2, 1) {};
    				\node [style=none] (9) at (2, 0.75) {};
    				\node [style=none] (10) at (3, 0.75) {};
    				\node [style=none] (11) at (3, 2) {};
    				\node [style=none, scale=1.2] (12) at (1.5, 1.75) {$\eta$};
    				\node [style=none, scale=1.2] (13) at (2.5, 0.1) {$\epsilon$};
    				\node [style=none] (14) at (0.75, 0.25) {$A$};
    				\node [style=none] (15) at (3.25, 1.75) {$A$};
    			\end{pgfonlayer}
    			\begin{pgfonlayer}{edgelayer}
    				\draw (6.center) to (7.center);
    				\draw [bend left=90, looseness=1.50] (7.center) to (8.center);
    				\draw (8.center) to (9.center);
    				\draw [bend right=90, looseness=1.50] (9.center) to (10.center);
    				\draw (10.center) to (11.center);
    			\end{pgfonlayer}
    		\end{tikzpicture} = 
            \begin{tikzpicture}[scale=1.2]
    		  \draw (0,2.5) -- (0,0);
    		\end{tikzpicture} ~~~~~~~~~~
    		\begin{tikzpicture}[scale=1.2]
    			\begin{pgfonlayer}{nodelayer}
    				\node [style=none] (6) at (3, 0) {};
    				\node [style=none] (7) at (3, 1) {};
    				\node [style=none] (8) at (2, 1) {};
    				\node [style=none] (9) at (2, 0.75) {};
    				\node [style=none] (10) at (1, 0.75) {};
    				\node [style=none] (11) at (1, 2) {};
    				\node [style=none, scale=1.2] (12) at (2.5, 1.75) {$\eta$};
    				\node [style=none, scale=1.2] (13) at (1.5, 0.1) {$\epsilon$};
    				\node [style=none] (14) at (3.25, 0.25) {$B$};
    				\node [style=none] (15) at (0.75, 1.75) {$B$};
    			\end{pgfonlayer}
    			\begin{pgfonlayer}{edgelayer}
    				\draw (6.center) to (7.center);
    				\draw [bend right=90, looseness=1.50] (7.center) to (8.center);
    				\draw (8.center) to (9.center);
    				\draw [bend left=90, looseness=1.50] (9.center) to (10.center);
    				\draw (10.center) to (11.center);
    			\end{pgfonlayer}
    		\end{tikzpicture} =
            \begin{tikzpicture}[scale=0.9]
    		  \draw (0,2.5) -- (0,0);
    		\end{tikzpicture} \]
		
    A symmetric {\bf $*$-autonomous category} is an LDC in which every object has a chosen dual object.
        
\end{frame}

\begin{frame}{Biclosed LDCs}

An LDC is:

\tcolor{$\ox$-closed} if for all $A:\X$, the functor $ - \otimes A : \X \to \X$  has a right adjoint.
	\[ \X(A \ox B, C) \cong \X(B, A \lollipop C) \quad \quad \color{teal}{\eval: A \ox (A \lollipop B) \to B} \]
\tcolor{$\tri$-coclosed} if for all $A: \X$, the functor $ - \tri A: \X \to \X$ has a left adjoint.
	\[ \X \left(A \slash C, B \right) \cong \X(A, B \tri C) \quad \quad \color{teal}{\coeval: A \to (A \slash B) \tri B} \]
\tcolor{biclosed} if $\X$ is both $\ox$-closed and $\tri$-coclosed.
    
\end{frame}

\begin{frame}{Poly is biclosed}
    
    The category \textcolor{magenta}{$\Poly$ is a biclosed isomix LDC} where, for any two polynomials $p, q$:
    
    \[ p \lollipop q := [p,q]  =  \prod_{P : p(1)} \sum_{\color{magenta} Q : q(1)} \prod_{d : q[Q]} \sum_{\color{magenta}d' : p[P]} y \]   
    
    \[ p \slash q := \coclose{q}{p}  = \sum_{P : p(1)} y^{ q \tri p[P]} = \sum_{P : p(1)} \prod_{\color{magenta} Q: q(1)} \prod_{d:q[Q]\to p[P]} y \] 
    
\end{frame}

\begin{frame}{Double closure maps (1)}

For any object $A:\X$ in a biclosed isomix LDC, there is a sort of a ``double closure'' map:

\[ \begin{tikzcd}[ampersand replacement=\&]
  A \ar[r, "{\color{magenta}\coeval}"] \ar[rrr, bend right = 15pt, "\Phi_A"'] \& [10pt]  \left(\coclose{[A, y]}{y} \tri [A,y] \right) \ox A \ar[r, "\color{magenta}\partial^R"] \&   \coclose{[A, y]}{y} \tri ([A,y]) \ox A) \ar[r, "\color{magenta}\eval"] \& \coclose{[A,y]}{y} 
\end{tikzcd} \]

\textcolor{teal}{Theorem:} If $A$ is right dual to $[A,y]$ with 
 \[ \epsilon = \eval: A \ox [A,y] \to y \]  
 then $\Phi_A$ has a retraction $\chi_A$,
 \[ \begin{tikzcd}[ampersand replacement=\&]
 	A \ar[r, "\Phi_A"] \ar[rr, bend right, equals] \& \coclose{[A,y]}{y} \ar[r, "\chi_A"] \& A
 \end{tikzcd} \]  
For the other direction, we need an extra condition: $ \coeval \then \partial^R \then \chi_A \then \eval = Id_{[A, y]}$.

\end{frame}

\begin{frame}{Double closure maps (2)}

For any object $a:\X$ in a biclosed isomix LDC there is a sort of a ``double closure'' map:

\[ \begin{tikzcd}[ampersand replacement=\&]
  \left[ \coclose{A}{ y },  y  \right] \ar[r, "\color{magenta}\coeval"] \ar[rrr, bend right = 15pt, "\Psi_A"']
\& \left[ \coclose{A}{ y },  y  \right] \ox \left( \coclose{A}{ y } \tri A \right) \ar[r, "\color{magenta}\partial^L"] 
\& \left( \left[ \coclose{A}{ y },  y  \right] \ox \coclose{A}{ y } \right) \tri A \ar[r, "\color{magenta}\eval"] 
\& A
\end{tikzcd} \]

\textcolor{teal}{Theorem:} If $A$ is left dual to $\coclose{A}{y}$ with
 \[ \eta = \coeval: y \to \coclose{A}{y} \tri A \]  
then $\Psi_A$ has a section $\Omega_a$:
\[ \begin{tikzcd}[ampersand replacement=\&]  
A \ar[ r, "\Omega_A"] \ar[rr, bend right, equals] \& \left[ \coclose{A}{y}, y \right] \ar[r, "\Psi_A"] \& A
\end{tikzcd} 
\]

\end{frame}



\begin{frame}{Double closure maps in Poly }

%\[ [Ay, y] \cong y^A \quad \quad \quad \quad \quad \quad \quad [y^A, y] \cong Ay \] 

%\vspace{1em}

\textcolor{teal}{Theorem:} A polynomial \textcolor{magenta}{$p = y^A$} for $A: \text{Set}$ if and only if \tcolor{$\Phi_p$ is the identity}:
\[ \id_p = \Phi_p : p \to \coclose{[p,y]}{y} \]
\textcolor{teal}{Corollary:}  For any set $A$, \tcolor{$Ay$ is left dual to $y^A$} with $\eta = \coeval$ and $\epsilon = \eval$
\[ \includegraphics[scale=0.08]{pics/dual2.jpeg}\]
\textcolor{teal}{Theorem:} A polynomial \textcolor{magenta}{$q = Ay$} for $A: \text{Set}$ if and only if \tcolor{$\Psi_q$ is the identity}.
\[ \id_q = \Psi_q : \left[ \coclose{q}{y}, y \right] \to q \]
We get the same corollary again!

\end{frame}


\begin{frame}{What are the linear duals in Poly?}

\textcolor{teal}{Theorem:} If a polynomial $p$ is left dual to $q$, then $p = Ay$ and $q = y^A$ for some $A : {\sf Set}$.  
\[ \includegraphics[scale=0.08]{pics/dual3.jpeg}\]
Thus, the only polynomials with duals are linear polynomials and representables!

\vspace{1em}

\textcolor{teal}{Theorem}: If $q \dual p$ and $p \dual q$, then $p = q = y$.

\vspace{0.25em}

Any polynomial which is both a left and a right dual of the same polynomial is trivial.

\end{frame}

\begin{frame}[noframenumbering,plain]
\begin{tikzpicture}[remember picture, overlay]
        \fill[black] (current page.south west) rectangle ([xshift=4cm]current page.north west);
    \end{tikzpicture}
    \begingroup  
        \flushleft
        {\fontfamily{qag}\selectfont\hspace{2 cm}\Large\bfseries\color{black}{Part III: The core of mix LDC}}\vspace{1em}\\
    \endgroup
\end{frame}

\begin{frame}{The core of a mix LDC}
    
    The {\bf core}\footnote{Richard Blute, Robin Cockett, Robert Seely, "Feedback for linearly dis- tributive categories: traces and fixpoints" (2000)} of a mix LDC is the full subcategory determined by objects $A$ for which the natural transformation is also an isomorphism:
    \[ \indep_{A,-}: A \ox - \to A\tri - \quad \quad \quad  \indep_{-, A}: - \ox A \to - \tri A  \]
    
     \vspace{0.25em}
    
    The core of a \textcolor{purple}{mix category} is closed to $\otimes$ and $\oplus$.
    
    \vspace{0.5em}
    
    The core of an \textcolor{purple}{isomix} LDC contains the monoidal units $\top$ and $\bot$.  
    
    \vspace{0.5em}
    
    The core of an isomix LDC is linearly equivalent to its underlying monoidal categories.
    
\end{frame}

\begin{frame}{Left and right core of mix LDCs}
    For a { mix ($\ox$-symmetric) LDC} $(\X, \otimes, \top, \tri, \bot)$, 
    
    \tcolor{Left core:} Full subcategory of objects $A: \X$ such that 
    \[ \color{magenta}{\indep_{A,-}: A \ox - \to A \tri -} \]
    \tcolor{Right core:}  Full subcategory of objects $B:\X$ such that 
    \[ \color{magenta}{\indep_{-, B}: - \ox B \to - \tri B} \]
    \textcolor{teal}{Lemma:} For an isomix LDC, the unit object is both in the left and the right cores. 
    
    In an isomix LDC, the left and the right core are compact LDCs.    
\end{frame}

\begin{frame}{Opposing cores}

    \tcolor{Opposing cores:} A mix LDC is said to have opposing cores if 
    \[ \star: \rCore(\X)^{\op} \to^{\cong} \lCore(\X) \]   
	 \textcolor{teal}{Examples:} Compact closed categories and Poly
	 	   
	 \vspace{0.75em}
		     
	    \textcolor{teal}{Lemma:} In Poly, a polynomial \tcolor{$p$ in the left core} if and only if $p \cong Ay$ for some $A:\Set$.
		
		\textcolor{teal}{Lemma:} In Poly, a polynomial \tcolor{$q$ in the right core} if and only if $q \cong y^B$ for some $B:\Set$.
		\[ \includegraphics[scale=0.1]{pics/boxes1.jpeg} \]
		
\end{frame}

\begin{frame}{Opposing cores of Poly }

\textcolor{teal}{Corllary:} The category $\Poly$ has opposing cores. 


\textcolor{teal}{Proof:} For any $A: \Set$,  \[ \left(y^A \right)^\star := Ay \]

For any $A,B: \Set$, and map $\varphi: y^B \to y^A$,
\[ \varphi^\star: Ay \to By ; \quad (\varphi^\star)_1 ~:=~ \varphi^\sharp \]

\textcolor{teal}{Corollary:} In Poly, a polynomial \tcolor{$p$ is left dual to $q$} if and only if \tcolor{$p$ is from the left core} and \tcolor{$q$ is from the right core}.
\[ \includegraphics[scale=0.08]{pics/boxes2.jpeg} \]
\end{frame}

\begin{frame}[noframenumbering,plain]
\begin{tikzpicture}[remember picture, overlay]
        \fill[black] (current page.south west) rectangle ([xshift=2cm]current page.north west);
    \end{tikzpicture}
    \begingroup  
        \flushleft
        {\fontfamily{qag}\selectfont\hspace{2 cm}\Large\bfseries\color{black}{Part IV: Linear monoids and linear comonoids}}\vspace{1em}\\
        \hspace{2 cm} Robin Cockett, Jurgen Koslowski and Robert Seely. \\ 
        \hspace{2.5 cm} {\em Introduction to Linear bicategories} (1999) \\
        \hspace{2 cm} Priyaa Varshinee Srinivasan. (PhD Thesis) \\
        \hspace{2.5 cm} {\em Dagger linear logic for categorical quantum mechanics} (2021) \\
    \endgroup
\end{frame}

\begin{frame}{Linear monoids in LDCs}

In an LDC, a {\bf linear monoid}\footnote{Robin Cockett, Jurgen Koslowski and Robert Seely.{Introduction to Linear bicategories} (1999)}, $A \linmonw B$, contains a:

\vspace{0.5em} 

- a $\ox$-monoid $(A, \mulmap{1.35}{white}: A \ox A \to A, ~\unitmap{1.35}{white}: \top \to A)$ 

\vspace{0.5 em}

 - cyclic duals, $A \dual B$ and $B \dual A$

\vspace{0.5em}

together producing a $\tri$-comonoid $(B, \comulmap{1.5}{white}: B \to B \tri B, \counitmap{1.5}{white}: B \to \bot)$
\[ 	(i)~~~	
		 \begin{tikzpicture}
		\begin{pgfonlayer}{nodelayer}
			\node [style=circle] (0) at (-2.75, 0.75) {};
			\node [style=none] (1) at (-3.25, 1.25) {};
			\node [style=none] (2) at (-2.75, 0.5) {};
			\node [style=none] (3) at (-2.25, 1.25) {};
			\node [style=none] (4) at (-1.5, 1.25) {};
			\node [style=none] (5) at (-1.5, -0) {};
			\node [style=none] (6) at (-3.25, 1.5) {};
			\node [style=none] (7) at (-1, 1.5) {};
			\node [style=none] (8) at (-1, -0) {};
			\node [style=none] (9) at (-3.75, 0.5) {};
			\node [style=none] (10) at (-3.75, 2.25) {};
			\node [style=none] (11) at (-4, 2) {$B$};
			\node [style=none] (12) at (-1.75, 0.25) {$B$};
			\node [style=none] (13) at (-0.75, 0.25) {$B$};
		\end{pgfonlayer}
		\begin{pgfonlayer}{edgelayer}
			\draw [in=150, out=-90, looseness=1.00] (1.center) to (0);
			\draw [in=-90, out=30, looseness=1.00] (0) to (3.center);
			\draw (0) to (2.center);
			\draw [bend left=90, looseness=1.50] (2.center) to (9.center);
			\draw (9.center) to (10.center);
			\draw (6.center) to (1.center);
			\draw [bend left=90, looseness=1.25] (6.center) to (7.center);
			\draw (7.center) to (8.center);
			\draw (4.center) to (5.center);
			\draw [bend right=90, looseness=1.25] (4.center) to (3.center);
		\end{pgfonlayer}
	\end{tikzpicture} = \begin{tikzpicture}
		\begin{pgfonlayer}{nodelayer}
			\node [style=circle] (0) at (-2, 0.75) {};
			\node [style=none] (1) at (-1.5, 1.25) {};
			\node [style=none] (2) at (-2, 0.5) {};
			\node [style=none] (3) at (-2.5, 1.25) {};
			\node [style=none] (4) at (-3.25, 1.25) {};
			\node [style=none] (5) at (-3.25, -0) {};
			\node [style=none] (6) at (-1.5, 1.5) {};
			\node [style=none] (7) at (-3.75, 1.5) {};
			\node [style=none] (8) at (-3.75, -0) {};
			\node [style=none] (9) at (-1, 0.5) {};
			\node [style=none] (10) at (-1, 2.25) {};
			\node [style=none] (11) at (-0.75, 2) {$B$};
			\node [style=none] (12) at (-3, 0.25) {$B$};
			\node [style=none] (13) at (-4, 0.25) {$B$};
		\end{pgfonlayer}
		\begin{pgfonlayer}{edgelayer}
			\draw [in=30, out=-90, looseness=1.00] (1.center) to (0);
			\draw [in=-90, out=150, looseness=1.00] (0) to (3.center);
			\draw (0) to (2.center);
			\draw [bend right=90, looseness=1.50] (2.center) to (9.center);
			\draw (9.center) to (10.center);
			\draw (6.center) to (1.center);
			\draw [bend right=90, looseness=1.25] (6.center) to (7.center);
			\draw (7.center) to (8.center);
			\draw (4.center) to (5.center);
			\draw [bend left=90, looseness=1.25] (4.center) to (3.center);
		\end{pgfonlayer}
	\end{tikzpicture} 
	~~~~~~~~~~~~ 
	(ii)~~~ 	
	\begin{tikzpicture}
		\begin{pgfonlayer}{nodelayer}
			\node [style=circle] (0) at (-0.75, 1.5) {};
			\node [style=none] (1) at (-0.75, 0.5) {};
			\node [style=none] (2) at (-1.75, 0.5) {};
			\node [style=none] (3) at (-1.75, 2.25) {};
			\node [style=none] (4) at (-2, 2) {$B$};
		\end{pgfonlayer}
		\begin{pgfonlayer}{edgelayer}
			\draw (0) to (1.center);
			\draw [bend left=90, looseness=1.50] (1.center) to (2.center);
			\draw (2.center) to (3.center);
		\end{pgfonlayer}
	\end{tikzpicture} = 
	\begin{tikzpicture}
		\begin{pgfonlayer}{nodelayer}
			\node [style=circle] (0) at (-2, 1.5) {};
			\node [style=none] (1) at (-2, 0.5) {};
			\node [style=none] (2) at (-1, 0.5) {};
			\node [style=none] (3) at (-1, 2.25) {};
			\node [style=none] (4) at (-0.75, 2) {$B$};
		\end{pgfonlayer}
		\begin{pgfonlayer}{edgelayer}
			\draw (0) to (1.center);
			\draw [bend right=90, looseness=1.50] (1.center) to (2.center);
			\draw (2.center) to (3.center);
		\end{pgfonlayer}
	\end{tikzpicture} 
 \] 
\end{frame}

\begin{frame}{Left and right linear monoids in LDCs}

In an LDC, a \tcolor{left linear monoid}, $A \linmonwl B$, contains a:

- a $\ox$-monoid $(A, \mulmap{1.35}{white}: A \ox A \to A, ~\unitmap{1.35}{white}: \top \to A)$ on the left dual, and

- a dual, $A \dual B$.
  
 The $\ox$-monoid structure on $A$ induces a $\tri$-comonoid structure on $B$ via the duality:
 \[ \begin{tikzpicture}
			\begin{pgfonlayer}{nodelayer}
				\node [style=none] (12) at (1.75, 0.25) {$B$};
				\node [style=none] (13) at (0.25, 0.25) {$B$};
				\node [style=circle] (17) at (1, 1.25) {};
				\node [style=none] (18) at (1, 2.5) {};
				\node [style=none] (19) at (0.5, 0) {};
				\node [style=none] (20) at (1.5, 0) {};
				\node [style=none] (21) at (1.25, 2.25) {$B$};
			\end{pgfonlayer}
			\begin{pgfonlayer}{edgelayer}
				\draw [in=-150, out=90, looseness=1.25] (19.center) to (17);
				\draw [in=90, out=-30, looseness=1.25] (17) to (20.center);
				\draw (17) to (18.center);
			\end{pgfonlayer}
		\end{tikzpicture} = \begin{tikzpicture}
		\begin{pgfonlayer}{nodelayer}
			\node [style=circle] (0) at (-2, 0.75) {};
			\node [style=none] (1) at (-1.5, 1.25) {};
			\node [style=none] (2) at (-2, 0.5) {};
			\node [style=none] (3) at (-2.5, 1.25) {};
			\node [style=none] (4) at (-3.25, 1.25) {};
			\node [style=none] (5) at (-3.25, -0) {};
			\node [style=none] (6) at (-1.5, 1.5) {};
			\node [style=none] (7) at (-3.75, 1.5) {};
			\node [style=none] (8) at (-3.75, -0) {};
			\node [style=none] (9) at (-1, 0.5) {};
			\node [style=none] (10) at (-1, 2.25) {};
			\node [style=none] (11) at (-0.75, 2) {$B$};
			\node [style=none] (12) at (-3, 0.25) {$B$};
			\node [style=none] (13) at (-4, 0.25) {$B$};
		\end{pgfonlayer}
		\begin{pgfonlayer}{edgelayer}
			\draw [in=30, out=-90, looseness=1.00] (1.center) to (0);
			\draw [in=-90, out=150, looseness=1.00] (0) to (3.center);
			\draw (0) to (2.center);
			\draw [bend right=90, looseness=1.50] (2.center) to (9.center);
			\draw (9.center) to (10.center);
			\draw (6.center) to (1.center);
			\draw [bend right=90, looseness=1.25] (6.center) to (7.center);
			\draw (7.center) to (8.center);
			\draw (4.center) to (5.center);
			\draw [bend left=90, looseness=1.25] (4.center) to (3.center);
		\end{pgfonlayer}
	\end{tikzpicture} 
 \quad \quad \quad \begin{tikzpicture}
		\begin{pgfonlayer}{nodelayer}
			\node [style=circle] (17) at (1, 0.25) {};
			\node [style=none] (18) at (1, 2.5) {};
			\node [style=none] (21) at (1.25, 2.25) {$a$};
		\end{pgfonlayer}
		\begin{pgfonlayer}{edgelayer}
			\draw (17) to (18.center);
		\end{pgfonlayer}
	\end{tikzpicture} = 	
	\begin{tikzpicture}
		\begin{pgfonlayer}{nodelayer}
			\node [style=circle] (0) at (-2, 1.5) {};
			\node [style=none] (1) at (-2, 0.5) {};
			\node [style=none] (2) at (-1, 0.5) {};
			\node [style=none] (3) at (-1, 2.25) {};
			\node [style=none] (4) at (-0.75, 2) {$B$};
		\end{pgfonlayer}
		\begin{pgfonlayer}{edgelayer}
			\draw (0) to (1.center);
			\draw [bend right=90, looseness=1.50] (1.center) to (2.center);
			\draw (2.center) to (3.center);
		\end{pgfonlayer}
	\end{tikzpicture}  \]
	
In an LDC, a \tcolor{right linear monoid}, $A \linmonwr B$, contains a:

- a $\ox$-monoid structure on the right dual $B$, and

- a dual, $A \dual B$.

 The $\ox$-monoid structure on $B$ induces a $\tri$-comonoid structure on $A$ via the duality.

\end{frame}

\begin{frame}{Left and right linear monoids in Poly}

The functor $A\mapsto Ay$ is strong monoidal $(\Set,1,\times)\to(\Poly,y,\otimes)$

The functor $A\mapsto y^A$ is strong monoidal $(\Set^\op,1,\times)\to(\Poly,y,\otimes)$.

\vspace{0.5em}

For any monoid $(M,*,u)$ in Set, $My \linmonwl y^M$ is a \tcolor{left linear monoid} because:

- $(\coeval, \eval): My \dual y^M$,

- The functor $M \mapsto My$ is (strong) monoidal preserves monoids. 

\vspace{0.75em}

For any set $A$, $My \linmonwr y^M$ is a \tcolor{right linear monoid} because:


- $(\coeval, \eval): My \dual y^M$,

- The functor $M \mapsto y^M$ is strong monoidal, hence preserves comonoids. Every set has a unique comonoid structure given by the diagonal map.

\end{frame}

\begin{frame}{Linear comonoids}

In an LDC, a {\bf linear comonoid}\footnote{Priyaa Varshinee Srinivasan. (PhD Thesis) { Dagger linear logic for categorical quantum mechanics} (2021)} $A \lincomonw B$, contains a:

\vspace{0.5em} 

- a \underline{$\ox$-comonoid} $(A, \comulmap{1.35}{white}: A \to A \ox A, ~\counitmap{1.35}{white}: A \to \top)$ 

\vspace{0.5 em}

 - cyclic duals, $A \dual B$ and $B \dual A$

\vspace{0.5em} 

\[ {\it (i)}~~~ \begin{tikzpicture}
		\begin{pgfonlayer}{nodelayer}
			\node [style=circle] (0) at (1.7, 2.75) {};
			\node [style=none] (1) at (1.2, 2) {};
			\node [style=none] (2) at (2.2, 2) {};
			\node [style=none] (3) at (1.7, 3.5) {};
			\node [style=none] (5) at (0.7, 2) {};
			\node [style=none] (6) at (-0.3, 2) {};
			\node [style=none] (7) at (2.7, 3.5) {};
			\node [style=none] (8) at (2.7, 1.25) {};
			\node [style=none] (9) at (2.95, 1.75) {$B$};
			\node [style=oa] (10) at (0.2, 3) {};
			\node [style=none] (11) at (0.2, 3.75) {};
			\node [style=none] (12) at (-0.25, 3.65) {$B \tri B$};
		\end{pgfonlayer}
		\begin{pgfonlayer}{edgelayer}
			\draw [in=-165, out=90, looseness=1.25] (1.center) to (0);
			\draw [in=90, out=-15, looseness=1.25] (0) to (2.center);
			\draw (0) to (3.center);
			\draw [bend left=90, looseness=1.75] (1.center) to (5.center);
			\draw [bend right=90] (6.center) to (2.center);
			\draw [bend right=90, looseness=1.25] (7.center) to (3.center);
			\draw (7.center) to (8.center);
			\draw [in=90, out=-45] (10) to (5.center);
			\draw [in=90, out=-135, looseness=1.25] (10) to (6.center);
			\draw (10) to (11.center);
		\end{pgfonlayer}
	\end{tikzpicture} = \begin{tikzpicture}
			\begin{pgfonlayer}{nodelayer}
				\node [style=circle] (0) at (0.75, 2.75) {};
				\node [style=none] (1) at (1.25, 2) {};
				\node [style=none] (2) at (0.25, 2) {};
				\node [style=none] (3) at (0.75, 3.5) {};
				\node [style=none] (5) at (1.75, 2) {};
				\node [style=none] (6) at (2.75, 2) {};
				\node [style=none] (7) at (-0.25, 3.5) {};
				\node [style=none] (8) at (-0.25, 1.25) {};
				\node [style=none] (9) at (-0.5, 1.75) {$B$};
				\node [style=oa] (10) at (2.25, 3) {};
				\node [style=none] (11) at (2.25, 3.75) {};
				\node [style=none] (12) at (2.75, 3.65) {$B \tri B$};
			\end{pgfonlayer}
			\begin{pgfonlayer}{edgelayer}
				\draw [in=-15, out=90, looseness=1.25] (1.center) to (0);
				\draw [in=90, out=-165, looseness=1.25] (0) to (2.center);
				\draw (0) to (3.center);
				\draw [bend right=90, looseness=1.75] (1.center) to (5.center);
				\draw [bend left=90] (6.center) to (2.center);
				\draw [bend left=90, looseness=1.25] (7.center) to (3.center);
				\draw (7.center) to (8.center);
				\draw [in=90, out=-135] (10) to (5.center);
				\draw [in=90, out=-45, looseness=1.25] (10) to (6.center);
				\draw (10) to (11.center);
			\end{pgfonlayer}
		\end{tikzpicture}		
	    ~~~~~~~~
	    {\it (ii)} ~~~ 
	    \begin{tikzpicture}
	    	\begin{pgfonlayer}{nodelayer}
	    		\node [style=none] (0) at (1.5, 3.5) {};
	    		\node [style=none] (2) at (2.5, 3.5) {};
	    		\node [style=none] (3) at (2.5, 1) {};
	    		\node [style=none] (4) at (2.75, 1.75) {$B$};
	    		\node [style=circle, scale=1.5] (5) at (0.75, 1.25) {};
	    		\node [style=none] (6) at (0.75, 1.25) {$\bot$};
	    		\node [style=none] (7) at (0.75, 4.25) {};
	    		\node [style=circle] (8) at (0.75, 2) {};
	    		\node [style=none] (9) at (1.5, 2.75) {};
	    		\node [style=circle] (14) at (1.5, 2.75) {};
	    	\end{pgfonlayer}
	    	\begin{pgfonlayer}{edgelayer}
	    		\draw [bend right=90, looseness=1.50] (2.center) to (0.center);
	    		\draw (2.center) to (3.center);
	    		\draw (5) to (7.center);
	    		\draw [dotted, in=-90, out=0, looseness=1.25] (8) to (9.center);
	    		\draw (0.center) to (14);
	    	\end{pgfonlayer}
	    \end{tikzpicture} = \begin{tikzpicture}
	    	\begin{pgfonlayer}{nodelayer}
	    		\node [style=none] (0) at (2, 3.5) {};
	    		\node [style=none] (2) at (1, 3.5) {};
	    		\node [style=none] (3) at (1, 1) {};
	    		\node [style=none] (4) at (0.75, 1.75) {$B$};
	    		\node [style=circle, scale=1.5] (5) at (2.75, 1.25) {};
	    		\node [style=none] (6) at (2.75, 1.25) {$\bot$};
	    		\node [style=none] (7) at (2.75, 4.25) {};
	    		\node [style=circle] (8) at (2.75, 2) {};
	    		\node [style=none] (9) at (2, 2.75) {};
	    		\node [style=circle] (14) at (2, 2.75) {};
	    	\end{pgfonlayer}
	    	\begin{pgfonlayer}{edgelayer}
	    		\draw [bend left=90, looseness=1.50] (2.center) to (0.center);
	    		\draw (2.center) to (3.center);
	    		\draw (5) to (7.center);
	    		\draw [dotted, in=-90, out=180, looseness=1.25] (8) to (9.center);
	    		\draw (0.center) to (14);
	    	\end{pgfonlayer}
	    \end{tikzpicture} \]
	    
	    We apply the same idea of left and right linear monoids to linear comonoids, and get similar examples in Poly.
	    
	    \vspace{0.5em}
	    
\end{frame}


\begin{frame}{Left and right linear comonoids}

In an LDC, a \tcolor{left linear comonoid}, $A \lincomonwl B$, contains a:

- a \underline{$\ox$-comonoid} on the left dual $A$, and

- a dual, $A \dual B$.

 The $\ox$-comonoid on $A$ induces a $\tri$-monoid on $B$. 
 
 In right linear comonoids, right duals carry the $\ox$-comonoid structure.
 
\textcolor{teal}{Examples:}
  
In Poly, for any monoid $(M,*,u)$ in Set, $My \lincomonwr y^M$ is a \tcolor{right linear comonoid}.


In Poly, for any set $A$, $Ay \lincomonwl y^A$ is a \tcolor{left linear comonoid}.

\end{frame}


\begin{frame}[noframenumbering,plain]

\begin{tikzpicture}[remember picture, overlay]
        \fill[black] (current page.south west) rectangle ([xshift=2cm]current page.north west);
    \end{tikzpicture}
    \begingroup  
        \flushleft
        {\fontfamily{qag}\selectfont\hspace{2 cm}\Large\bfseries\color{black}{Part V: Linear bialgebras}}\vspace{1em} \\
        \hspace{2 cm} Priyaa Varshinee Srinivasan. \\
        \hspace{2.5 cm} {\em Dagger linear logic for categorical quantum mechanics} (Thesis 2021) \\
    \endgroup
\end{frame}

\begin{frame}{Bialgebra in symmetric monoidal categories}

In a symmetric monoidal category $(\X, \ox, I)$, a {\bf bialgebra} consists of a

- a monoid $\monoid{A}$

- a comonoid $\comonoidb{A}$

satisfying the following rules:

\[ \begin{tikzpicture}[yscale=-1]
	\begin{pgfonlayer}{nodelayer}
		\node [style=circle] (0) at (0, 4) {};
		\node [style=circle, fill=black] (1) at (0, 5) {};
		\node [style=none] (2) at (-0.5, 3) {};
		\node [style=none] (3) at (0.5, 3) {};
	\end{pgfonlayer}
	\begin{pgfonlayer}{edgelayer}
		\draw [bend right, looseness=1.25] (0) to (2.center);
		\draw [bend left, looseness=1.25] (0) to (3.center);
		\draw (1) to (0);
	\end{pgfonlayer}
\end{tikzpicture} = \begin{tikzpicture}[yscale=-1]
	\begin{pgfonlayer}{nodelayer}
		\node [style=circle, fill=black] (1) at (0, 5) {};
		\node [style=none] (3) at (0, 3) {};
		\node [style=circle, fill=black] (4) at (0.75, 5) {};
		\node [style=none] (5) at (0.75, 3) {};
	\end{pgfonlayer}
	\begin{pgfonlayer}{edgelayer}
		\draw (3.center) to (1);
		\draw (5.center) to (4);
	\end{pgfonlayer}
\end{tikzpicture} \quad \quad \quad \quad \begin{tikzpicture}
	\begin{pgfonlayer}{nodelayer}
		\node [style=circle, fill=black] (0) at (0, 4) {};
		\node [style=circle] (1) at (0, 5) {};
		\node [style=none] (2) at (-0.5, 3) {};
		\node [style=none] (3) at (0.5, 3) {};
	\end{pgfonlayer}
	\begin{pgfonlayer}{edgelayer}
		\draw [bend right, looseness=1.25] (0) to (2.center);
		\draw [bend left, looseness=1.25] (0) to (3.center);
		\draw (1) to (0);
	\end{pgfonlayer}
\end{tikzpicture} = \begin{tikzpicture}
	\begin{pgfonlayer}{nodelayer}
		\node [style=circle] (1) at (0, 5) {};
		\node [style=none] (3) at (0, 3) {};
		\node [style=circle] (4) at (0.75, 5) {};
		\node [style=none] (5) at (0.75, 3) {};
	\end{pgfonlayer}
	\begin{pgfonlayer}{edgelayer}
		\draw (3.center) to (1);
		\draw (5.center) to (4);
	\end{pgfonlayer}
\end{tikzpicture} \quad \quad \quad \quad \begin{tikzpicture}
	\begin{pgfonlayer}{nodelayer}
		\node [style=circle] (1) at (0, 5) {};
		\node [style=circle, fill=black] (2) at (0, 3) {};
	\end{pgfonlayer}
	\begin{pgfonlayer}{edgelayer}
		\draw (1) to (2);
	\end{pgfonlayer}
\end{tikzpicture} = \id_I  \quad \quad \quad \quad \begin{tikzpicture}
	\begin{pgfonlayer}{nodelayer}
		\node [style=circle, fill=black] (0) at (0, 5.25) {};
		\node [style=circle, fill=black] (1) at (1.25, 5.25) {};
		\node [style=circle] (2) at (0, 3.75) {};
		\node [style=circle] (3) at (1.25, 3.75) {};
		\node [style=none] (4) at (0, 3) {};
		\node [style=none] (5) at (1.25, 3) {};
		\node [style=none] (6) at (0, 6) {};
		\node [style=none] (7) at (1.25, 6) {};
		\node [style=none] (8) at (1.25, 3) {};
	\end{pgfonlayer}
	\begin{pgfonlayer}{edgelayer}
		\draw (1) to (2);
		\draw (0) to (3);
		\draw [bend right=45] (0) to (2);
		\draw [bend left=45] (1) to (3);
		\draw (3) to (8.center);
		\draw (2) to (4.center);
		\draw (6.center) to (0);
		\draw (7.center) to (1);
	\end{pgfonlayer}
\end{tikzpicture} = \begin{tikzpicture}
	\begin{pgfonlayer}{nodelayer}
		\node [style=none] (4) at (0, 3) {};
		\node [style=none] (6) at (0, 6) {};
		\node [style=none] (7) at (1.5, 6) {};
		\node [style=none] (8) at (1.5, 3) {};
		\node [style=circle, fill=black] (9) at (0.75, 4) {};
		\node [style=circle] (10) at (0.75, 5.25) {};
	\end{pgfonlayer}
	\begin{pgfonlayer}{edgelayer}
		\draw [in=90, out=-165] (9) to (4.center);
		\draw [in=90, out=-15] (9) to (8.center);
		\draw (9) to (10);
		\draw [in=255, out=15] (10) to (7.center);
		\draw [in=285, out=165] (10) to (6.center);
	\end{pgfonlayer}
\end{tikzpicture} \]  

\end{frame}

\begin{frame}{Linear bialgebra in symmetric LDC}

	In a symmetric LDC, a {\bf linear bialgebra}\footnote{Priyaa Varshinee Srinivasan. (PhD Thesis) { Dagger linear logic for categorical quantum mechanics} (2021)} consists of:
	
		- a linear monoid, $A \linmonw B$, and
		
		- a linear comonoid, $A \lincomonb B$,

	such that:
	
		- $\bialg{A}$ is a $\ox$-bialgebra, and 
		
		- $\bialgb{B}$ is a $\tri$-bialgebra. 
		
	\vspace{1em}
		
	We can get linear bialgebras in half symmetric case too! 
	
	\vspace{0.5em}
	
	We apply the same left-right trick again! You get the idea!
	
	\vspace{0.5em}

\end{frame}

\begin{frame}{{Left and right} linear bialgebras in $\ox$-symmetric LDC}

	In a \tcolor{$\ox$-}symmetric LDC, a \tcolor{left linear bialgebra} consists of:
	
		- a \tcolor{left} linear monoid, $A \linmonwl B$, and
		
		- a \tcolor{left} linear comonoid, $A \lincomonwl B$ \quad (denoted below by $\bullet$),

	such that $\bialg{A}$ is a $\ox$-bialgebra. 
		
	\vspace{1em}				

	In a \tcolor{$\ox$-}symmetric LDC, a \tcolor{right linear bialgebra} consists of:
	
		- a \tcolor{right} linear monoid, $A \linmonwr B$, and
		
		- a \tcolor{right} linear comonoid, $A \lincomonwr B$ \quad (denoted below by $\bullet$),

	such that $\bialg{B}$ is a $\ox$-bialgebra. 
	
\end{frame}

\begin{frame}{Left and right linear bialgebras in Poly}

\vspace{0.5em}

\textcolor{teal}{Lemma:} If $(M, *, e)$ is a monoid in $\Set$ then the dual \tcolor{$(\coeval, \eval): My \dual y^M$} in Poly carries all four structures:

\begin{center}
\begin{tabular}{  c|c } 
 left linear monoid & left linear comonoid \\ 
 $My \linmonwl y^M$ & $My \lincomonwl y^M$ \\
 \hline
 &  \\
 right linear  monoid &  right linear comonoid \\ 
  $My \linmonwr y^M$ & $My \lincomonwr y^M$ \\

\end{tabular}
\end{center}

Moreover, we get a \tcolor{left linear bialgebra} (1st row) and a \tcolor{right linear bialgebra} (2nd row).

\vspace{0.5em}

\textcolor{teal}{Corollary:} In Poly, we have that $My \dual y^M$ is both a left and right linear bialgebra if and only if $M$ is a monoid in Set.

\end{frame}

\begin{frame}{Summary}

- All normal duoidal categories are isomix LDCs.

- (Poly, $\ox$, $\tri$, y) is an $\ox$-symmetric isomix LDC.  

- In Poly, for any set $A$, $Ay$ is left dual to $y^A$.

- In Poly, the only objects with duals are linear polynomials and represenatables.

- Poly has left- and right- linear- monoids, comonoids, and bialgebras.

\vspace{1em}

\tcolor{A bugging question:}

What is the connection between the cores and the biclosure property? Is there one?

\end{frame}

\end{document}